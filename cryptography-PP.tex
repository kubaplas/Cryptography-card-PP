\documentclass[10pt,titlepage]{article}
\usepackage{graphicx}
\usepackage{graphics}
\usepackage{epsfig}
\usepackage{amsmath}
\usepackage{amssymb}
\usepackage{amsthm}
\usepackage{booktabs}
\usepackage{stmaryrd}
\usepackage{url}
\usepackage{graphicx}
\usepackage{longtable}
\usepackage[figuresright]{rotating}
\usepackage[cp1250]{inputenc}
\usepackage[T1]{fontenc}
\usepackage[english]{babel}
\usepackage{pslatex}
\usepackage{ulem}
\usepackage{lipsum}
\usepackage{listings}
\usepackage{url}
\usepackage{color}
\usepackage[left=3cm,top=3cm,right=3cm]{geometry} 
\usepackage{bchart}
\usepackage{datapie}
\usepackage{calc}
\usepackage{ifthen}
\usepackage{tikz}
\usepackage{hyperref}

\definecolor{szary}{gray}{0.6}% jasnoszary

\setlength{\textwidth}{400pt}
\lstset{numbers=left,
			numberstyle=\tiny, 
			basicstyle=\scriptsize\ttfamily, 
			breaklines=true, 
			captionpos=b, 
			tabsize=2}

\usepackage[ruled,vlined,linesnumbered]{algorithm2e}
\newcommand{\RR}{\mathbb{R}}
\newcommand{\NN}{\mathbb{N}}
\newcommand{\QQ}{\mathbb{Q}}
\newcommand{\ZZ}{\mathbb{Z}}
\newcommand{\TAB}{\hspace{0.50cm}}
\newcommand{\IFF}{\leftrightarrow}
\newcommand{\IMP}{\rightarrow}
\newcommand{\slice}[4]{
  \pgfmathparse{0.5*#1+0.5*#2}
  \let\midangle\pgfmathresult

  % slice
  \draw[thick,fill=black!10] (0,0) -- (#1:1) arc (#1:#2:1) -- cycle;

  % outer label
  \node[label=\midangle:#4] at (\midangle:1) {};

  % inner label
  \pgfmathparse{min((#2-#1-10)/110*(-0.3),0)}
  \let\temp\pgfmathresult
  \pgfmathparse{max(\temp,-0.5) + 0.8}
  \let\innerpos\pgfmathresult
  \node at (\midangle:\innerpos) {#3};
}
\makeindex






\begin{document}

\pagestyle{empty}

\begin{titlepage}
\vspace*{\fill}
\begin{center}
\begin{picture}(300,510)
  \put( 10,520){\makebox(0,0)[l]{\large \bf \textsc{Faculty of Fundamental Problems of Technology}}}
  \put( 10,500){\makebox(0,0)[l]{\large \bf \textsc{Wroclaw University of Technology}}}
  \put( 20,380){\makebox(0,0)[l]{\Huge  \bf \textsc{Card - Terminal}}}
  \put( 20,340){\makebox(0,0)[l]{\Huge  \bf \textsc{Authorization Protocol}}}	
	\put( 20,300){\makebox(0,0)[l]{\Huge  \bf \textsc{Protection Profile}}}
  \put(100,240){\makebox(0,0)[l]{\large     \textsc{Andrzej Rybczak}}}
  \put(100,220){\makebox(0,0)[l]{\large     \textsc{Jakub Plaskonka}}}
  \put(100,200){\makebox(0,0)[l]{\large     \textsc{Bartlomiej Paciorek}}}
  \put(100,180){\makebox(0,0)[l]{\large     \textsc{Mateusz Platek}}}


  \put(100,-80){\makebox(0,0)[bl]{\large \bf \textsc{Wroclaw 2013}}}
\end{picture}
\end{center}
\vspace*{\fill}
\end{titlepage}

\tableofcontents

\pagestyle{headings}
\newpage
\section{Protection Profile Introduction }


\section{Security Problem Definition}
\subsection{Assets}
Place for assets.
\begin{table}[!ht]
    \begin{tabular}{|l|l|l|}
    \hline
			Asset name & Comment & Protection Goal \\ \hline
			Card & ~ & ~ \\ \hline
			Terminal & ~ & ~ \\ \hline
			~ & ~ & ~ \\ \hline
			~ & ~ & ~ \\ \hline
    \end{tabular}
\end{table}



and rest of actors

\subsection{Threats}

To chyba wypelnimy razem, trzeba wymienic wektory atakow.
\subsection{Assumptions}
\subsection{Organization Security Policies}

\section{Security Objectives}
\subsection{Security Objectives for the TOE}
\subsection{Security Objectives for the Operational Enviroment}

\section{Security Requirements}
\section{Conformance Claims}

\bibliographystyle{plain}

\bibliography{bibliography}

\end{document}
