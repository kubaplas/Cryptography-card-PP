\documentclass[10pt,titlepage]{article}
\usepackage{graphicx}
\usepackage{graphics}
\usepackage{epsfig}
\usepackage{amsmath}
\usepackage{amssymb}
\usepackage{amsthm}
\usepackage{booktabs}
\usepackage{url}
\usepackage{graphicx}
\usepackage{longtable}
\usepackage[figuresright]{rotating}
\usepackage[cp1250]{inputenc}
\usepackage[T1]{fontenc}
\usepackage[english]{babel}
\usepackage{pslatex}
\usepackage{ulem}
\usepackage{lipsum}
\usepackage{listings}
\usepackage{url}
\usepackage{color}
\usepackage[left=3cm,top=3cm,right=3cm]{geometry} 
\usepackage{bchart}
\usepackage{datapie}
\usepackage{calc}
\usepackage{ifthen}
\usepackage{tikz}
\usepackage{hyperref}
 \usepackage{array,multirow}
 \usepackage{xcolor,colortbl}

\definecolor{light-gray}{gray}{0.95}

\setlength{\textwidth}{400pt}
\lstset{numbers=left,
			numberstyle=\tiny, 
			basicstyle=\scriptsize\ttfamily, 
			breaklines=true, 
			captionpos=b, 
			tabsize=2}

\usepackage[ruled,vlined,linesnumbered]{algorithm2e}
\newcommand{\RR}{\mathbb{R}}
\newcommand{\NN}{\mathbb{N}}
\newcommand{\QQ}{\mathbb{Q}}
\newcommand{\ZZ}{\mathbb{Z}}
\newcommand{\TAB}{\hspace{0.50cm}}
\newcommand{\IFF}{\leftrightarrow}
\newcommand{\IMP}{\rightarrow}
\newcommand{\slice}[4]{
  \pgfmathparse{0.5*#1+0.5*#2}
  \let\midangle\pgfmathresult

  % slice
  \draw[thick,fill=black!10] (0,0) -- (#1:1) arc (#1:#2:1) -- cycle;

  % outer label
  \node[label=\midangle:#4] at (\midangle:1) {};

  % inner label
  \pgfmathparse{min((#2-#1-10)/110*(-0.3),0)}
  \let\temp\pgfmathresult
  \pgfmathparse{max(\temp,-0.5) + 0.8}
  \let\innerpos\pgfmathresult
  \node at (\midangle:\innerpos) {#3};
}
\makeindex






\begin{document}

\section*{Simple but longterm card-terminal authorization protocol based on one time passwords - sketch of protocol}

 \begin{table}[!ht]
 \centering
	\begin{tabular}{| p{4.55cm} |  p{4.55cm} | p{4.55cm}| }
 \hline
 Card & Transmission  & Terminal \\\hline
 & & Sends Hello Messsage and asks card for authentication.\\
 & $1. \leftarrow Hello Message$ & \\
 Generates the challenge nonce (NC). &  &  \\ 
 &   $2. NC \rightarrow $&  \\ 
 &  & Prepare the challenge nonce (NR). \newline Respond on NC nonce with AR value.\newline Cipher both nonce and value with KeyStream. \\ 
 & $3. \leftarrow KeyStream1 \oplus NR + KeyStream2 \oplus AR $&\\
 Deciphers the response, verifies the challenge AR and responds with AC value.\newline Also sends current State of card.\newline Everything is ciphered with usage of the same KeyStream. & & \\
 & $4. AC \oplus KeyStream3\: ,\: State \oplus KeyStream4 \rightarrow $& \\
 & & Verify the AC challange. \newline Verify the state. \newline Send new state.\\
 &$5. \leftarrow NewState \oplus KeyStream5$ &\\
 Save the new state. & &\\
 \hline

 \end{tabular}
 \end{table}
 
 To detect the attemp of relay attack the terminal is gathering the informations about the delays between each card response (in miliseconds).
 If the sum of delays is higher than given safe value, the card is revoked on the end of authentication procedure.
 


\end{document}